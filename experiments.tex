\section{Experimental validation}\label{sec:experiments}

In this section, we validate the utility of the CLEAVE framework through scalability measurements of a networked control system running on a WiFi link.
With this, we aim to answer questions about the ability of CLEAVE to provide relevant and accurate metrics on the performance of such setups as \emph{system load} increases.
In this context, we will focus on the load as it manifests on the network link in terms of the reliability of the link and the associated latencies. 
We adopt this constrained view of load due to the characteristics of NCS, where controllers are usually relatively lightweight in terms of computing resources, leaving the network link as the principal bottleneck in the system.

Our experimental setup is depicted in \cref{fig:expsetup}.
A number between 1 and 12 of CLEAVE Plants running on dedicated light-weight general purpose computing devices (in this case, Raspberry Pi 4's) connect wirelessly to a WiFi Access Point (AP).
This AP in turn connects via Ethernet to a host on which the CLEAVE Dispatcher and Controllers are executed.

\begin{figure}
    \centering
    \includegraphics[width=3cm]{example-image-a}
    \caption{Experiment setup}
    \label{fig:expsetup}
\end{figure}