\section{Conclusion}\label{sec:conclusion}

The issue of repeatable and scalable benchmarks has been largely glossed over in \ac{NCS} literature, as existing experimental research studies tend to implement \emph{ad-hoc} solutions.

In this work, we aim to tackle this issue by presenting a fully software-based framework for repeatable, reproducible, and easily scalable \ac{NCS} benchmarking with a particular focus on edge deployment.
\todo[inline]{The conclusion needs to firmly argue WHY we achieved what we set out to do.}
Our framework, \ac{CLEAVE}, allows for the easy implementation of emulated \acp{NCS} on real networks.
It is fully parametrizable, extendable, and built with industry-standard edge technologies and paradigms in mind.
We validate the utility of this tool through an example use case showcasing the ability of the framework to extract relevant metrics relating to the stability of the control system, as well as on the performance of the underlying network link.
We believe \ac{CLEAVE} represents an important step towards enabling inexpensive and low-complexity scalable research for real-world deployment of edge-bound \acp{NCS}.

There is still, however, work to be done.
We are extending the number of plant and controller implementations on the framework, with the goal of creating an open library of \acp{NCS} to share with the community.
At the moment, the interactions of \ac{CLEAVE} and tools such as Docker are still quite superficial.
Our goal is to achieve a much tighter integration, e.g.\ by providing the toolkit as pre-packaged container images.
Finally, the validity of the results obtained by the framework will have to be verified through more thorough, realistic scenarios than what we have been able to show in this work.
In particular, we intend to perform large-scale experimentation targetting 5G cellular deployments, as this technology is set to become the backbone of edge networks in the near future.
