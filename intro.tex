\section{Introduction}\label{sec:intro}

\todo[inline]{change this}

With the advent of the Internet of Things and Industry 4.0, the number and applications of cyber physical systems is ever increasing [1].
Cyber-Physical Systems (CPS) are formed by an integration of computational and physical processes, namely, it comprises of actuators and sensors connected by virtue of a computational process [2]. A Networked Control System (NCS) is composed of a CPS wherein multiple networked actuators and sensors form a part of the same automatic control system [3].
Zhang et al. in [4] proposed a control communication co-design approach wherein the stabilization of an NCS in which telemetry between the sensors and actuators of the plant with a remote controller utilizing a shared communication medium is discussed.
Zoppi et al. in [5] proposed a platform called NCSBench, to be used for reproducible benchmarking in NCS.
Their methodology utilizes joint knowledge of control, computation, and communication.
In their work various architectural elements and the corresponding delays associated with the NCS are modelled.
Multiple experimental parameters and certain observable Key Performance Indicators (KPIs) are defined and utilized in the implementation.
This work however utilizes a physical LEGO [6] based plant for the implementation, preventing instantaneous changes in the plant characteristics and their nature, in terms of flexibility.
We overcome this issue by proposing a completely virtual plant allowing for unparalleled flexibility in changing the plant model and characteristic features of the experiments.
This work experiments over Ethernet and Wireless Local Area Network (WLAN) based networks.
We perform experiments with similar network configurations and additionally benchmark the number of plants which can be simultaneously sustained by the controller at the same time over varying physical sampling frequencies and tick rates.

\todo[inline]{change this}