\PassOptionsToPackage{pdftex}{graphicx}
\documentclass[sigconf,9pt,natbib=false]{acmart}

% begin packages
% ==============
\usepackage[utf8]{inputenc}
\usepackage[T1]{fontenc}
% \usepackage[pdftex]{graphicx}
\graphicspath{{./images/}{./plots/}}
\DeclareGraphicsExtensions{.pdf,.jpeg,.png}
\usepackage{amsmath}
\usepackage{url}
\usepackage[inline]{enumitem}
\usepackage{todonotes}
\usepackage{caption}
\usepackage{subcaption}
\pagenumbering{gobble}
\usepackage[newfloat]{minted}
\SetupFloatingEnvironment{listing}{name=List.,within=none}
\captionsetup[table]{justification=centerlast,
                     labelsep=newline,
                     font=sf,
                     textfont=footnotesize}
\usepackage{booktabs}
\usepackage{xcolor}
\usepackage[detect-weight=true,detect-family=true,binary-units=true,list-units=single,range-units=single]{siunitx}
\usepackage[nolist]{acronym}
\begin{acronym}
  \acro{CPS}{Cyber-Physical System}
  \acro{NCS}{Networked Control System}
  \acroindefinite{NCS}{an}{a}
  \acro{LAN}{Local-Area Network}
  \acro{WLAN}{Wireless Local-Area Network}
  \acro{KPI}{Key Performance Indicator}
  \acro{WPAN}{Wireless Personal Area Network}
  \acro{CSMA/CD}{Carrier-Sense Multiple Access with Collision Detection}
  \acro{CLEAVE}{ControL bEnchmArking serVice on the Edge}
  \acro{OS}{Operating System}
  \acro{UDP}{User Datagram Protocol}
  \acro{TCP}{Transmission Control Protocol}
  \acro{RMS}{Root Mean Square}
  \acro{RTT}{Round-Trip Time}
  \acro{CI}{Confidence Interval}
  \acro{AP}{Access Point}
  \acro{API}{Application Programming Interface}
  \acro{SSF}{Swedish Foundation for Strategic Research}
  \acro{TECoSA}{Trustworthy Edge Computing Systems and Applications}
  \acro{ABC}{Abstract Base Class}
  \acroplural{ABC}[ABCs]{Abstract Base Classes}
\end{acronym}

% references
% \usepackage[
%   style=numeric-comp,
%   sorting=none,
%   sortcites,
%   hyperref,
%   mincitenames=1,
%   maxcitenames=2,
%   maxbibnames=2,
%   minbibnames=1,
%   citestyle=numeric-comp, % for [1, 2] instead of [1], [2]
%   backend=bibtex
% ]{biblatex}
% \bibliography{bibliography.bib}
% \AtBeginBibliography{\small}
% \AtEveryBibitem{\clearfield{day}}
% \AtEveryBibitem{\clearfield{isbn}}
% \AtEveryBibitem{\clearfield{url}}
% \AtEveryBibitem{\clearfield{series}}
% \AtEveryBibitem{\clearlist{location}}
% \AtEveryBibitem{\clearfield{doi}}

% for dealing with ACM's stupid bibtex requirements
% \usepackage{biblatex2bibitem}
\usepackage[nosort,compress]{cite}

\usepackage{orcidlink}
\usepackage[all]{hypcap}
\usepackage[capitalize,nameinlink,noabbrev]{cleveref}
% \crefname{subfigure}{Subfigure}{Subfigures}
% \Crefname{subfigure}{Subfigure}{Subfigures}
\crefname{listing}{Listing}{Listings}
\Crefname{listing}{Listing}{Listings}

\hypersetup{
  hidelinks,
  colorlinks=true,
  allcolors=black,
  pdfstartview=Fit,
  breaklinks=true
}

\title[CLEAVE]{CLEAVE:\@ Scalable and Edge-Native Benchmarking of\\{Networked Control Systems}}

%authorlist
\author{Manuel {Olguín Muñoz}}
\orcid{0000-0002-3383-2335}
\email{molguin@kth.se}
\affiliation{%
\institution{KTH Royal Institute of Technology}%
% \department[0]{Division of Information Science and Engineering}%
% \department[1]{School of EECS}%
\city{Stockholm}%
\country{Sweden}%
}

\author{Neelabhro Roy}
\orcid{0000-0002-5777-7780}
\email{nroy@kth.se}
\affiliation{%
\institution{KTH Royal Institute of Technology}%
% \department[0]{Division of Information Science and Engineering}%
% \department[1]{School of EECS}%
\city{Stockholm}%
\country{Sweden}%
}

\author{James Gross}
\orcid{0000-0001-6682-6559}
\email{jamesgr@kth.se}
\affiliation{%
\institution{KTH Royal Institute of Technology}%
% \department[0]{Division of Information Science and Engineering}%
% \department[1]{School of EECS}%
\city{Stockholm}%
\country{Sweden}%
}

% top matter
% \acmConference[EdgeSys'22]{}{}{}  % TODO
\settopmatter{printacmref=true, printccs=true, printfolios=true}
\setcopyright{none}

\begin{CCSXML}
<ccs2012>
    <concept>
        <concept_id>10003033.10003079.10003082</concept_id>
        <concept_desc>Networks~Network experimentation</concept_desc>
        <concept_significance>300</concept_significance>
        </concept>
    <concept>
        <concept_id>10003033.10003079.10011672</concept_id>
        <concept_desc>Networks~Network performance analysis</concept_desc>
        <concept_significance>300</concept_significance>
        </concept>
    <concept>
        <concept_id>10002944.10011123.10010912</concept_id>
        <concept_desc>General and reference~Empirical studies</concept_desc>
        <concept_significance>300</concept_significance>
        </concept>
    <concept>
        <concept_id>10002944.10011123.10010916</concept_id>
        <concept_desc>General and reference~Measurement</concept_desc>
        <concept_significance>300</concept_significance>
        </concept>
    <concept>
        <concept_id>10002944.10011123.10011131</concept_id>
        <concept_desc>General and reference~Experimentation</concept_desc>
        <concept_significance>500</concept_significance>
        </concept>
    <concept>
        <concept_id>10002944.10011123.10011130</concept_id>
        <concept_desc>General and reference~Evaluation</concept_desc>
        <concept_significance>300</concept_significance>
        </concept>
    <concept>
        <concept_id>10002944.10011123.10011674</concept_id>
        <concept_desc>General and reference~Performance</concept_desc>
        <concept_significance>500</concept_significance>
        </concept>
    <concept>
        <concept_id>10010520.10010553.10010559</concept_id>
        <concept_desc>Computer systems organization~Sensors and actuators</concept_desc>
        <concept_significance>500</concept_significance>
        </concept>
    <concept>
        <concept_id>10010520.10010553.10010554.10010556</concept_id>
        <concept_desc>Computer systems organization~Robotic control</concept_desc>
        <concept_significance>100</concept_significance>
        </concept>
    <concept>
        <concept_id>10010520.10010521.10010537.10003100</concept_id>
        <concept_desc>Computer systems organization~Cloud computing</concept_desc>
        <concept_significance>300</concept_significance>
        </concept>
    <concept>
        <concept_id>10010520.10010521.10010537.10010538</concept_id>
        <concept_desc>Computer systems organization~Client-server architectures</concept_desc>
        <concept_significance>300</concept_significance>
        </concept>
  </ccs2012>
\end{CCSXML}

\ccsdesc[300]{Networks~Network experimentation}
\ccsdesc[300]{Networks~Network performance analysis}
\ccsdesc[300]{General and reference~Empirical studies}
\ccsdesc[300]{General and reference~Measurement}
\ccsdesc[500]{General and reference~Experimentation}
\ccsdesc[300]{General and reference~Evaluation}
\ccsdesc[500]{General and reference~Performance}
\ccsdesc[500]{Computer systems organization~Sensors and actuators}
\ccsdesc[100]{Computer systems organization~Robotic control}
\ccsdesc[300]{Computer systems organization~Cloud computing}
\ccsdesc[300]{Computer systems organization~Client-server architectures}

\copyrightyear{2022} 
\acmYear{2022} 
\setcopyright{rightsretained} 
\acmConference[EdgeSys'22]{5th International Workshop on Edge Systems, Analytics and Networking }{April 5--8, 2022}{RENNES, France}
\acmBooktitle{5th International Workshop on Edge Systems, Analytics and Networking (EdgeSys'22), April 5--8, 2022, RENNES, France}\acmDOI{10.1145/3517206.3526272}
\acmISBN{978-1-4503-9253-2/22/04}

\begin{abstract}
    \acfp*{NCS} have long been a subject of theoretical research, however, little work has been devoted to repeatable and scalable experimentation.
    The advent of edge computing and the subsequent proliferation of \acsp*{NCS} has quickly led this to become a key issue that needs to be addressed.

    In this work, we present a novel, completely software-based approach to this challenge.
    Our core idea rests on a framework for the emulation of physical plants which communicate with controllers implemented in software over a real network.
    This approach is furthermore designed and built for the Edge, using a general-purpose programming language and with full compatibility with industry-standard containerization solutions.
    We validate this framework using an initial implementation of an inverted pendulum \acs*{NCS}.
    Our results showcase the utility of the tool as a repeatable, extensible, and scalable solution to \acs*{NCS} performance evaluation.
\end{abstract}


\begin{document}
\maketitle
\renewcommand{\shortauthors}{{Olguín Muñoz} et al.}

\section{Introduction}\label{sec:intro}

\begin{figure*}
    \centering
    \includegraphics[width=.8\textwidth]{CLEAVE_NCS_structure}
    \caption{
        Structure of an emulated \acl*{NCS} in \acs*{CLEAVE}.
    }\label{fig:cleave:ncs:struct}
\end{figure*}

The number and applications of \acp{CPS}~\cite{Rajkumar2010CPS} --- i.e.\ systems in which a real, physical mechanism is controlled by a computer --- have exploded in recent years.
However, this rapid increase in adoption has mostly been limited to industrial contexts.
Although \acp{CPS} present huge opportunities for all facets of society, they have yet to reach our daily lives in any relevant scale due to their stringent operational requirements.
This is about to change, however, as with the advent of novel wireless communication technologies as well as networking paradigms, such as cellular 5G and Edge Computing~\cite{Satya2017Emergence}, consumer-grade \acp{CPS} will be made possible.
These technologies meet two key requirements of \acp{CPS}: real-time capabilities (through extremely low end-to-end latencies), and context- and locality-awareness, and will most likely become the backbone of \ac{CPS} in the future.

A subcategory of \acp{CPS} which stands to particularly benefit from the adoption of the above technologies is \acp{NCS}~\cite{Gupta2010NCSOverview}, a type of \ac{CPS} wherein multiple networked actuators and sensors form a part of the same automatic control system.
Depending on the physical system being controlled, \acp{NCS} can have stringent timing and reliability requirements for the communication between components that conventional cloud paradigms and cellular networks are unable to meet~\cite{Wan2020Efficient}.

Due to their potential advantages for industrial and commercial settings, a large body of work exists dedicated to the modelling and performance characterization of \acp{NCS}~\cite{Zhang2013Survey,Zhang2016Survey}.
They improve and flexibilize existing control systems by allowing for the distribution of control functions over and across networks.
This allows for, e.g.\ centralized coordination, control, and monitoring of potentially hundreds of devices.

Most of the literature concerning \acp{NCS}, however, follows a theoretical approach, and only a small fraction of it deals with experimental studies.
The inherent inter-domain nature of \acp{NCS}, which intertwines knowledge from the fields of communications, computing, and control theory, coupled with the complexity of experimental platforms make these kinds of studies hard to perform.

One approach to experimental research in \acp{NCS} uses setups in which the complete system is built on top of real hardware.
This approach is employed in the works of, for instance, \textcite{Li2014Wireless,Baumann2018LowPower,Cuenca2019UAV}; in all of these, the authors implement and validate their results on physical testbeds.
Conversely, other studies choose to instead use completely \emph{simulated} \ac{NCS} setups.
\textcite{Wu2012NPC,Chen2015synccontrol,Ma2019DynamicSched} are examples of works in which the authors have opted for such completely virtualized approaches.
These studies often employ combinations of physical and network simulation tools to try to capture the complex dynamics of \acp{NCS}.
Finally, a number of experimental studies instead employ \emph{virtualized} approaches, in which either
\begin{enumerate*}[itemjoin={{; }}, itemjoin*={{; or }}]
    \item a real network interacts with a simulated or emulated control system~\cite{Wang2020VoltageControl}
    \item an emulated or simulated network interacts with a real control system~\cite{Natale2004InvPendEthernet}.
\end{enumerate*}

As evidenced above, experimental research in \acp{NCS} includes a large variety of heterogeneous hardware and software platforms, as well as methodologies and \aclp*{KPI}.
This, in turn, leads to hardware, software, and methodology fragmentation, as different studies tend to prefer approaches more favored in their respective communities.
Furthermore, existing studies tend to focus on individual aspects and components of a system, thus producing results which do not provide a complete image of the \ac{NCS}.
This has caused a gap in knowledge pertaining to the reproducibility and comparison of experimental studies on these systems.

\textcite{Zoppi2020NCSBench} made the first (and to the best of our knowledge, the only) attempt at tackling this challenge with their \emph{NCSbench} platform.
\emph{NCSbench} is an open-source \ac{NCS} benchmarking platform, built on top of the LEGO\textregistered{}\ Mindstorms EV3 Core Set\texttrademark{}\ platform, with stated goals of reproducibility and ease of use.
It achieves this thanks to its low-cost, accessibility, and ease of reconfiguration.

Although \emph{NCSbench}~\cite{Zoppi2020NCSBench} is a relevant and important step towards the reproducibility of \ac{NCS} benchmarks, we believe it still falls short on some aspects.
In particular, we argue the reliance of the tool on specific hardware for the Plant hampers the potential of the platform.
The LEGO Mindstorms platform, although accessible and flexible, still fundamentally limits the physical systems that can be benchmarked to the kind of systems that can be physically implemented on it.
Furthermore, it is \emph{not} scalable, and thus unsuitable for studies of larger, distributed \acp{NCS}.

In this paper, we present the first fully-software-based framework for scalable and repeatable benchmarking of edge-native \ac{NCS}.
As the Edge computing paradigm starts to be adopted by industry, more and more variations of it have started to appear in literature.
``Near'', ``far'', ``core'', and ``telco'' edge, among others, are all terms which describe variations of the original concept and which are becoming ubiquitous in new research.
While the core idea of Edge computing is widely accepted as fundamental for pervasive \acp{NCS} in general, understanding the strenghts and weaknesses of such different edge concepts is of paramount importance.

Our framework, which we name \ac{CLEAVE}, aims to simplify the repeatable and scalable benchmarking of such systems. 
It is built on top of the \emph{Python 3.8} language, making it highly extensible and able to harness the hundreds-of-thousands of already existing user-provided libraries and packages.
Additionally, it is compatible with \emph{containerization} technologies such as \emph{Docker}\footnote{Docker Engine: \url{https://www.docker.com/}}, making it suitable for automated deployment and benchmarking on industry-standard cloud and (more importantly) edge setups using container orchestration technologies.

The rest of this paper is structured as follows.
\cref{sec:approach} presents the design principles of the framework.
In \cref{sec:experiments}, we present a series of experiments which validate the utility, flexibility, and repeatability of \ac{CLEAVE}.
Finally, in \cref{sec:conclusion} we conclude this paper and briefly discuss future work.
\pagebreak
\section{The CLEAVE Framework}\label{sec:cleave}

CLEAVE (\emph{ControL bEnchmArking serVice on the Edge}) is our framework for quick, robust, and repeatable benchmarking of networked control systems.
CLEAVE tackles this challenge through a number of key design decisions that we will discuss in the following.

\begin{description}[wide]
    \item[Emulation approach.] 
    Existing benchmarking platforms for NCS either take a fully simulated approach --- running plant, network, and controller inside a completely simulated environment --- or one requiring an actual physical plant that interacts with a real network and controller.
    CLEAVE walks the line between these two approaches, employing a strategy based on having a real-time emulated representation of a Plant interact with the real network and controller.
    \item[Plant- and Controller-agnosticism.] 
    CLEAVE makes as few assumptions as possible about the inner workings of both the Plant and the Controller. 
    The only requirements for a system to be able to be emulated in CLEAVE are
    \begin{enumerate*}[itemjoin*={{ and }}]
        \item that its behavior can be described in a discrete-time fashion;
        \item that its state and the actions performed on it can be described by a finite number of scalar and/or vector variables.
    \end{enumerate*} 


    This agnosticism is achieved through careful abstraction of the the system into a well-defined API based around the following components:
    \begin{itemize}
        \item A \texttt{State} object, which implements the discrete-time evolution of the physical system being controlled.
        This object also holds a number of named properties that can be measured or actuated upon.
        \item An arbitrary number of \texttt{Sensor} objects which measure named properties of the \texttt{State}, and potentially transform them (for instance, to add noise).
        \item A \texttt{Controller} object which receives values of named properties processed by the \texttt{Sensor} objects and returns new values for the named properties of the \texttt{State} that can be actuated upon.
        \item Finally, an arbitrary number of \texttt{Actuator} objects which receive the values generated by the \texttt{Controller} and process them before passing them on the \texttt{State}.
    \end{itemize}
    Users implement their desired behavior by extending the abstract base classes defining these components provided by the framework.
    
\end{description}

\todo[inline]{Fill in core concepts}
\todo[inline]{Link to repo}

\subsection{General architecture}

The general architecture of CLEAVE centers around two components:

\begin{enumerate}
    \item the Plant, corresponding to the emulated physical system running on some arbitrary client device;
    \item the Controller Service running on some arbitrary cloudlet, which receives samples of the state of the Plant and produces control commands;
\end{enumerate}

These components are connected through the real network 

In the following, we will briefly discuss the design and implementation of these components.

\subsubsection{Plant}

As 

\subsubsection{Controller Service}
\subsection{Network}
\section{Experimental validation}\label{sec:experiments}

In this section, we demonstrate the utility of \ac{CLEAVE} through a series of experiments on an emulated \ac{NCS} running on a number of client devices and a cloudlet.
We aim to answer questions relating to the ability of our framework to provide accurate and repeatable measurements of the performance of \acp{NCS} deployed on edge computing infrastructure.

This section is structured in two parts.
\Cref{ssec:expsetup} details the experimental setup and describes the experiments performed.
\Cref{ssec:results} then presents and discusses the numerical results.

\subsection{Experimental Setup}\label{ssec:expsetup}

We detail our experimental edge setup in \cref{fig:cleave:expsetup,tab:hardware}.
It consists of \num{10} Raspberry Pi 4B  clients connected wirelessly to an IEEE 802.11n \ac{AP}; connected to the Ethernet backbone of this \ac{AP} is a general-purpose Cloudlet.
On top of this physical architecture we configure a Docker Swarm\footnote{Swarm mode overview: \url{https://docs.docker.com/engine/swarm/}} managed centrally from the Cloudlet.
For each experimental scenario we deploy a number of control loops inside this Swarm, executing plant containers exclusively in the Raspberry Pi clients (up to a single plant per client) and controller containers in the Cloudlet.
Plants and controllers communicate using the \ac{UDP} over an overlay network which sits on top of and abstracts away the real network configuration.
Additionally, to obtain baseline results without the stochastic effects of the network, we employ a secondary ``local-only'' setup, in which plants and controllers are executed co-located on the Cloudlet.

% Please add the following required packages to your document preamble:
% \usepackage{booktabs}
% \usepackage{graphicx}
\begin{table}[]
    \centering
    \caption{Hardware used in the experiments.}\label{tab:hardware}
    \resizebox{\columnwidth}{!}{%
    \begin{tabular}{@{}llrrrl@{}}
    \toprule
    \multicolumn{1}{c}{\textbf{}} &
      \multicolumn{1}{c}{\textbf{CPU}} &
      \multicolumn{1}{c}{\textbf{\begin{tabular}[c]{@{}c@{}}Freq\\ {[}\si{\giga\hertz}{]}\end{tabular}}} &
      \multicolumn{1}{c}{\textbf{\begin{tabular}[c]{@{}c@{}}Core\\ Count\end{tabular}}} &
      \multicolumn{1}{c}{\textbf{\begin{tabular}[c]{@{}c@{}}RAM\\ {[}\si{\giga\byte}{]}\end{tabular}}} &
      \multicolumn{1}{c}{\textbf{\begin{tabular}[c]{@{}c@{}}Operating\\ System\end{tabular}}} \\ \midrule
    \textbf{Cloudlet} &
      \begin{tabular}[c]{@{}l@{}}Intel\textregistered{} Core\texttrademark{}\\ i7-8700\end{tabular} &
      \num{3.2} &
      \num{6} &
      \num{32} &
      \begin{tabular}[c]{@{}l@{}}Ubuntu Server \\20.04 LTS \\Kernel v5.4.0\\\ \end{tabular} \\
    \textbf{Client} &
      \begin{tabular}[c]{@{}l@{}}Cortex-A72\\ (ARM v8)\end{tabular} &
      \num{2.0} &
      \num{4} &
      \num{8} &
      \begin{tabular}[c]{@{}l@{}}Ubuntu Server \\20.04 LTS \\Kernel v5.4.0\end{tabular} \\ \bottomrule
    \end{tabular}%
    }
\end{table}

\begin{figure}
    \centering
    \includegraphics[width=.95\columnwidth]{images/CLEAVE_experiment_setup}
    \caption{
        The setup used for our experimentation. 
        % Containerized versions of the core CLEAVE emulation components are deployed inside a Docker Swarm Overlay Network spanning \num{10} Raspberry Pi 4B clients connected to a single Cloudlet over an IEEE 802.11n \ac{AP}.
    }\label{fig:cleave:expsetup}
\end{figure}

\begin{figure*}[h]
    \centering
    \begin{subfigure}[t]{.5\textwidth}
        \centering
        \includegraphics[width=.9\textwidth]{single_loop_toppled}
        % \todo[inline]{Check consistency??}
        \caption{}%
        \label{fig:single:topple}
    \end{subfigure}%
    \begin{subfigure}[t]{.5\textwidth}
        \centering
        \includegraphics[width=.9\textwidth]{single_loop_rms}
        % \todo[inline]{Mark fully toppled experiments with an X.}
        \caption{}\label{fig:single:rms}
    \end{subfigure}%
    \caption[caption]{
        Stability metrics for the single-loop scenarios.
        \labelcref{fig:single:topple} shows the fraction of plants that toppled, per experimental setup.
        \labelcref{fig:single:rms} shows the mean \ac{RMS} value of the angle with respect to the vertical axis in logarithmic scale; the red line indicates \( y = 180 \).
        Error bars indicate \SI{95}{\percent} \acp{CI} in both plots.
        }%
    \label{fig:single:stability}
\end{figure*}

\begin{figure}
    \centering
    \includegraphics[width=.9\columnwidth]{single_loop_rtts}
    \caption{
        Mean latency due to network and processing for both single-loop scenarios.
        Error bars indicate \SI{95}{\percent} \acp{CI}.
    }\label{fig:single:rtt}
\end{figure}

\begin{figure*}[t]
    \centering
    \begin{subfigure}[h]{.25\textwidth}
        \centering
        \includegraphics[width=.9\textwidth]{video_topple_frac}
        \caption{}\label{fig:video:toppled}
    \end{subfigure}%
    \begin{subfigure}[h]{.25\textwidth}
        \centering
        \includegraphics[width=.9\textwidth]{video_angle_rms}
        \caption{}\label{fig:video:rms}
    \end{subfigure}%
    \begin{subfigure}[h]{.25\textwidth}
        \centering
        \includegraphics[width=.9\textwidth]{video_drop_frac}
        \caption{}\label{fig:video:drop}
    \end{subfigure}%
    \begin{subfigure}[h]{.25\textwidth}
        \centering
        \includegraphics[width=.9\textwidth]{video_rtt}
        \caption{}\label{fig:video:rtt}
    \end{subfigure}%
    \caption{
        Results for the realistic scenario.
        \labelcref{fig:video:toppled} shows the fraction of repetitions of each scenario in which \emph{at least} one plant failed to maintain stability and toppled.
        \labelcref{fig:video:rms} shows the \ac{RMS} for the pendulum angles for each scenario, only considering data from plants that did not topple.
        \labelcref{fig:video:drop} shows the fraction of \ac{UDP} datagrams dropped, averaged over all plants and repetitions per scenario.
        \labelcref{fig:video:rtt} shows the measured end-to-end plant-side \ac{RTT}, averaged over all plants and repetitions per scenario.
        Error bars indicate \SI{95}{\percent} \ac{CI} in all plots.
    }\label{fig:video:results}
\end{figure*}

\subsubsection{Single Loop Baseline Scenarios}

We first run a series of single loop scenarios with varying parametrization of the \ac{NCS}, intended as initial baselines to evaluate the accuracy of the \ac{CLEAVE} framework and showcase its flexibility.

For these scenarios, we vary:
\begin{enumerate*}[itemjoin={{; }}, itemjoin*={{; and }}]
    \item the sampling rate of the Plant state, setting it to \SIlist[list-final-separator={, or }]{10;20;40;60}{\hertz}
    \item the responsiveness of the Controller, adding fixed delays of \SIlist[list-final-separator={, or }]{0;25;50}{\milli\second} after the processing of each sample.
\end{enumerate*}

We repeat each combination of these parameters at least \num{10} times, for both the networked and ``local-only'' setups; experiments with sampling rates \( \geq \) \SI{20}{\hertz} and artificial delays \( \geq \) \SI{25}{\milli\second} were repeated an additional \num{20} times for better statistical significance.
Each repetition lasts for \SI{5}{\minute}, during which we collect detailed data on both the state of the controlled system as well as on the data sent over the network.

Scenarios are executed automatically in batches using a simple Python script which interacts with Docker through the widely adopted \emph{docker-py}\footnote{Docker SDK for Python: \url{https://docker-py.readthedocs.io/en/stable/}} library.
This is \ac{CLEAVE}'s first advantage over existing frameworks; it is designed with cloud and edge technology and paradigms in mind, making integration with existing systems straightforward.

\subsubsection{Realistic Scenario with Network Resource Contention}

Next, we run a multi-loop scenario to validate the utility of \ac{CLEAVE} in a more realistic setting where network resources are shared with video stream traffic.
Video analytics is one of the main proposed use cases for edge computing~\cite{Ananthanarayanan2017Analytics,Yi2017Analytics,Wang2018Analytics}, and thus we foresee edge \ac{NCS} deployments being deployed in parallel with such applications in the future.

In this scenario, we deploy \num{6} control loops on the experimental setup depicted in \cref{fig:cleave:expsetup}.
On the remaining \num{4} clients we run the \emph{iperf3}\footnote{iperf3: \url{https://iperf.fr/}} traffic load generator, each generating \SI[per-mode=symbol]{6.5}{\mega\bit\per\second} of uplink \ac{UDP} traffic.
This emulates the load generated on the network by \SI{1080}{p} Full-HD video streaming, originating from the clients and terminating in the cloudlet.
We execute this scenario with \ac{NCS} plant sampling rates of \SIlist{20;40;60}{\hertz}.
Each sampling rate configuration is run for \SI{5}{\minute}, and then repeated \num{30} times to obtain statistical significance.
Once again, repetitions of this scenario are executed automatically in batches using a simple Python script and \emph{docker-py}.

\subsection{Results}\label{ssec:results}

The results presented below provide valuable insights on both system limits and on the chosen \ac{NCS} itself, as well as on the capabilities of \ac{CLEAVE}.

\subsubsection{Single Plant Scenario}

We begin with a simple analysis of the stability of the Plant, as \ac{CLEAVE} allows for straightforward and repeatable collection and posterior processing of relevant metrics.  
\cref{fig:single:stability} shows the results relating to the stability of the plants in the single-loop scenarios:
\cref{fig:single:topple} shows the fraction of plants that toppled in each scenario, and \cref{fig:single:rms} shows the average \ac{RMS} for the absolute pendula angles.
Toppled plants were identified simply by analysis of the emulation data; instances whose pendulum angles reached values above a threshold\footnote{This value depends on the parametrization of the controller; in this case it corresponds to \SI{0.35}{\radian}.} were marked as ``toppled''.

As expected, higher sampling rates tend to correlate with better quality of control.
% (the \SI{40}{\hertz}, \SI{50}{\milli\second} scenario has a slightly inconsistent behavior that we attribute to randomness and the relatively short duration of the experiments).
At higher sampling rates the system was able to reach stability at higher \acp{RTT}.
These initial results already hint at interesting consequences for \ac{NCS} deployments on the edge.
For instance, it is clear from \cref{fig:single:stability} that network delays can, to a certain extent, be compensated for by increasing the sampling rate of the system.
A corollary of this is, conversely, that at lower network latencies \acp{NCS} are able to stabilize at lower sampling rates.
Adaptive sampling might thus be a viable method for optimizing \ac{NCS} resource utilization at the edge.

For completeness, we also showcase \ac{CLEAVE}'s ability to obtain network performance data from the experiments.
\cref{fig:single:rtt} shows average latency due to network and processing (excluding synthetic delays) for both single-loop scenarios.
Packet losses were below \SI{0.2}{\percent} for all parametrizations of the single-loop scenario over the network, and \SI{0}{\percent} for all parametrizations of the local-only setup.

\subsubsection{Realistic Scenario}

\cref{fig:video:results} shows a summary of the results obtained from the realistic scenario.
These results are interesting in their counter-intuitiveness.
Conventional wisdom would lead us to think that higher sampling rates are always better for the stability of control systems; however, \cref{fig:video:toppled,fig:video:rms} clearly show this not to be the case for \ac{NCS} in resource-constrained scenarios.
\SI{60}{\hertz} was the least stable configuration, with at least one pendulum toppling in around \SI{16}{\percent} of the repetitions, and mean pendulum angle \ac{RMS} circa \num{3} times that of the \SI{40}{\hertz} scenario.
\SI{40}{\hertz} was in turn the second worst configuration --- although it presented no toppled pendula, average angle \ac{RMS} doubled that of the \SI{20}{\hertz} setup.

We can see an explanation for these behaviors in \cref{fig:video:drop,fig:video:rtt}.
Whereas both the \SIlist{20;40}{\hertz} setups show losses well below \SI{5}{\percent}, the \SI{60}{\hertz} scenario shows an average of around \SI{13}{\percent} of datagrams lost.
The differences in \acp{RTT} results are equally telling; \acp{RTT} for the \SI{60}{\hertz} scenario were on average approx.\ \num{3} times those for the \SIlist{20;40}{\hertz} setups.

%(circa \SI{41}{\milli\second} versus \SIrange[range-phrase=--]{13}{15}{\milli\second}).

These results stem from the contention for network resources, and hint at important trade-offs system designers will have to take into consideration when designing and developing \acp{NCS} for deployment on the edge.
The edge will be \emph{multi-tenant} and \emph{multi-instance}. 
\ac{NCS} deployments will have to be designed with shared resources in mind, and given the complexity of these systems, experimental tools like \ac{CLEAVE} will be key for their succesful adoption and massification.

\section{Conclusion}\label{sec:conclusion}

The issue of repeatable and scalable benchmarks has been largely glossed over in \ac{NCS} literature; existing research is mostly focused on \emph{theory}, and the small fraction of experimental research studies available tend to implement \emph{ad-hoc} solutions.
To the best of our knowledge, the work by \textcite{Zoppi2020NCSBench} is the only available framework for \ac{NCS} benchmarking focused on repeatability and reproducibility.

In this work, we aim to tackle this issue by presenting a fully software-based framework for repeatable, reproducible, and easily scalable \ac{NCS} benchmarking with a particular focus on edge deployment.
Our framework, \ac{CLEAVE}, is fully parametrizable and built with industry-standard edge technologies and paradigms in mind.
We validate the utility of this tool through a series of scenarios, from which we are able to extract relevant metrics relating to the stability of the control system, as well as on the perfomance of the underlying network link.
\todo[inline]{Talk a bit more about results?}
We believe \ac{CLEAVE} represents an important step towards enabling inexpensive and low-complexity scalable research for real-world deployment of edge-bound \acp{NCS}.

There is still, however, work to be done.
To start with, the current implementation of \ac{CLEAVE} only includes a single plant-controller pair.
We envision creating an open library of plants and controllers to share with the community.
\ac{CLEAVE}'s integration with industry-standard tools and frameworks such as Docker will be refined.
At the moment, these interactions are still quite superficial.
Our goal is to achieve a much tighter integration, for instance by providing the complete toolkit as pre-packaged container images.
Finally, the validity of the results obtained by the framework will have to be verified through more thorough and realistic scenarios than what we have been able to show in this work.
In particular, we intend to eventually perform large-scale experimentation targetting 5G cellular deployments, as this technology is set to become the backbone of edge networks in the near future.

% Benchmarking human-in-the-loop applications is hard, given
% their tight interaction with human users who complicate the scaling
% and repeatability of experiments. In this paper, we have presented
% a benchmarking suite for this type of applications, called Edge-
% Droid 1.0, capable of cutting out the need for users in performance
% evaluations. We achieve this by employing pre-recorded sensory
% input traces which we play over the network to the real applica-
% tion backend, employing a parameterized user model to react to
% feedback. We demonstrate its utility through a series of use case
% scenarios, from which we are able to extract metrics regarding la-
% tency both in regards to the application itself and the hardware
% stack. We believe the EdgeDroid 1.0 suite thus represents an impor-
% tant rst step towards enabling inexpensive and low-complexity
% large-scale research on the scaling limits of this type of applications,
% a requirement for wide adoption of the technology.




% Nonetheless, there is still future work to be done.
% The user model presented in this paper is only preliminary, and
% we are currently conducting research in characterizing user behav-
% ior when interacting with WCA applications in order to present
% a more complete model in the future. As mentioned in Section 4,
% in EdgeDroid 1.0, our model is that of a user who does not suer
% any of the shortcomings of real human users such as annoyance, fa-
% tigue, frustration, nausea. Rather, EdgeDroid 1.0 models a perfectly
% stoic user who is like an automaton and responds in a precisely
% reproducible and deterministic manner to the same system stimu-
% lus every time. Of course, no real human user is an automaton. In
% the future, we envision creating many versions of EdgeDroid (i.e.,
% EdgeDroid 2.0, EdgeDroid 3.0, etc.) that embody more human-like
% user models that more accurately emulate attributes such as those
% mentioned above. Experimental validation of these human-like user
% models via user studies will be an important part of our future work.
% We are also working on expanding the benchmarking suite to
% also work rst with other types of Wearable Cognitive Assistance,
% and later with other categories of human-in-the-loop applications.
% Other types of WCA we will consider in future iterations of the tool
% include real-time task-assistance WCA applications (such as the
% Ping-Pong application described in [7]), which don’t have a linear
% task model like task-guidance WCA and have tighter latency bounds
% and context- and information-providing WCA applications, for
% instance, applications which recognize faces and provide relevant
% social-media information related to that person. The latter also do
% not have a linear task model, but present more lax latency bounds.
\section*{Acknowledgements}\label{sec:acks}
\todo[inline]{TODO: James? Funding? Project course students?}
% \printbibliography{}
% \printbibitembibliography{}

\begin{thebibliography}{99}
  %
  \bibitem{Rajkumar2010CPS}
  Ragunathan Rajkumar et al. “Cyber-physical systems: The next computing revolution”. In: \emph{Proceedings of the Design Automation Conference}. 2010, pp. 731–736.
  %
  \bibitem{Satya2017Emergence}
  Mahadev Satyanarayanan. “The Emergence of Edge Computing”. In: \emph{Computer} 50.1 (2017), pp. 30–39.
  %
  \bibitem{Gupta2010NCSOverview}
  Rachana Ashok Gupta and Mo-Yuen Chow. “Networked Control System: Overview and Research Trends”. In: \emph{IEEE Transactions on Industrial Electronics} 57.7 (2010), pp. 2527–2535. %
  \bibitem{Wan2020Efficient}
  Shaohua Wan et al. “Efficient computation offloading for {Internet of Vehicles} in edge computing-assisted 5G networks”. In: \emph{The Journal of Supercomputing} 76.4 (2020), pp. 2518–2547. %
  \bibitem{Baumann2018LowPower}
  Dominik Baumann et al. “Evaluating Low-Power Wireless Cyber-Physical Systems”. In: \emph{Proceedings of the First IEEE Workshop on Benchmarking Cyber-Physical Networks and Systems (CPSBench’18)}. 2018, pp. 13–18.
  %
  \bibitem{Cuenca2019UAV}
  Ángel Cuenca et al. “Periodic Event-Triggered Sampling and Dual-Rate Control for a Wireless Networked Control System With Applications to {UAVs}”. In: \emph{IEEE Transactions on Industrial Electronics} 66.4 (2019), pp. 3157–3166.
  %
  \bibitem{Ma2019DynamicSched}
  Yehan Ma et al. “Optimal Dynamic Scheduling of Wireless Networked Control Systems”. In: \emph{Proceedings of the 10th ACM/IEEE International Conference on Cyber-Physical Systems}. Association for Computing Machinery, 2019, 77–86.
  %
  \bibitem{Wang2020VoltageControl}
  Yu Wang et al. “Inverter-Based Voltage Control of Distribution Networks: A Three-Level Coordinated Method and Power Hardware-in-the-Loop Validation”. In: \emph{IEEE Transactions on Sustainable Energy} 11.4 (2020), pp. 2380–2391.
  %
  \bibitem{Natale2004InvPendEthernet}
  O.R. Natale et al. “Inverted pendulum stabilization through the {Ethernet} network, performance analysis”. In: \emph{Proceedings of the 2004 American Control Conference}. Vol. 6. 2004, 4909–4914 vol.6.
  %
  \bibitem{Zoppi2020NCSBench}
  Samuele Zoppi et al. “{NCSbench}: Reproducible Benchmarking Platform for Networked Control Systems”. In: \emph{Proceedings of the {17th} IEEE Annual Consumer Communications Networking Conference (CCNC’20)}. 2020, pp. 1–9.
  %
  \bibitem{Olguin2019EdgeDroid}
  Manuel Osvaldo J. Olguín Muñoz et al. “{EdgeDroid: An Experimental Approach to Benchmarking Human-in-the-Loop Applications}”. In: \emph{Proceedings of the 20th International Workshop on Mobile Computing Systems and Applications}. ACM, 2019, pp. 93–98.
  %
  \bibitem{Ananthanarayanan2017Analytics}
  Ganesh Ananthanarayanan et al. “Real-Time Video Analytics: The Killer App for Edge Computing”. In: \emph{Computer} 50.10 (2017), pp. 58–67.
  %
  \bibitem{Yi2017Analytics}
  Shanhe Yi et al. “LAVEA: Latency-Aware Video Analytics on Edge Computing Platform”. In: \emph{Proceedings of the Second ACM/IEEE Symposium on Edge Computing}. Association for Computing Machinery, 2017.
  %
  \bibitem{Wang2018Analytics}
  Junjue Wang et al. “Bandwidth-Efficient Live Video Analytics for Drones Via Edge Computing”. In: \emph{Proceedings of the Third IEEE/ACM Symposium on Edge Computing (SEC)}. Association for Computing Machinery, 2018, pp. 159–173.
\end{thebibliography}

\end{document}